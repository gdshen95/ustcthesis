\chapter{总结与展望}
大数据的发展是由于当今的互联网时代面临的信息爆炸越来越严重,如何从纷杂的信息中筛选出对用户有用的信息,是服务/商品提供商面临的重要问题。推荐系统的出现,是服务平台为了最大程度满足用户的需求与喜好,增加自身收益的系统。传统的单机系统,已经不能适应大数据环境下的应用场景,我们需要通过分布式计算来满足对于大规模计算的需求。Spark平台的出现,较好地解决了大数据时代对于迭代计算的要求。

在我的毕设设计中,较好地完成了基于Spark平台采用协同过滤算法的推荐系统,并对系统的可扩展性进行了一定的研究与优化。具体来说,基于Spark平台我们使用交替最小二乘法实现了基于矩阵分解的隐式因子模型,并且针对多组数据分析了结果;基于Spark提供的原语操作,我们实现了基于邻域的用户-用户相似关系、物品-物品相似关系的推荐算法,并且针对算法中会遇到的计算瓶颈问题采用了下采样的降低了计算复杂度。

未来我们可以考虑对基于邻域的协同过滤算法中使用不同的相似性关系进行研究,分析出不同的相似关系对于推荐系统的影响。同时可以将协同过滤与内容过滤进行结合起来,进行混合推荐。

