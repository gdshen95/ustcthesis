\chapter{结构设计}
\section{数据选择}
我们选择了由grouplens提供了MovieLens,数据主要主要划分四个部分。
\begin{itemize}
    \item 100k,这个部分主要是5分制,导致推荐结果较差。
    \item 1M, 5分制。
    \item 10M, 有半分了
    \item 20M, 有半分+1
\end{itemize}

\subsection{数据预处理部分}
groupLens组织上上个世纪90年代开始积累数据,由于中间经过4次比较大的变动,导致提供的数据并不完全按照一个标准收集,我们在实验中忽略这个问题,并在这个问题显示出影响的时候,予以指出。

标准提供的数据包含timestamp,由于我们不需要timestamp,在预处理过程将其去除。预处理过程我们可以在单机上完成,并不需要分布式。

\section{Spark架构搭建}
前面提及到,spark框架由master node和slave node组成,在我们搭建的spark集群中,使用了一台master node和两台slave node进行搭建。如下图
% todo 加一个我们的spark部署的图