\chapter{实验结果分析}
实验完成了TOPN推荐算法,并对提出的每种推荐算法进行评估分析。所采用的评估分析方法采用了rmse(root mean square error)方法。
\subsection{Root Mean Square Error}
均方根误差
$$rmse = \sqrt{\frac{\sum_{i=1}^{n}(r_{pre} - r_{test})^2}{n}}$$
根据公示,我们可以进行数据分析,得到如下结论。
\section{ALS算法推荐结果分析}
ALS算法运行结果分析图,我们先对100k的数据进行处理,使用100K的数据中较好的数据,来进行训练,得到较好的参数后,用来对更大量的数据进行训练。我们需要调整的参数主要有三个,maxiter、rank、和regParam。

通过我们训练好的参数,我们对比四个数据集的推荐效果。

\section{基于领域关系的算法推荐结果分析}
\section{两种算法推荐结果对比}
ALS算法随着输入数据的增加,推荐效果越来越好,在数据量较小的时候,推荐效果有时会弱于朴素的推荐算法。基于领域的推荐算法在数据量较小时就有不错的推荐效果,但是扩展性不如ALS算法,即使我们通过采样的方法使得数据量进行下降,但是因为采用了协同过滤算法中,不包含商品本身的信息,使得我们可能需要计算在真实场景中完全不相关的物品之间的相似性,这可能会导致一定的问题。

%补图