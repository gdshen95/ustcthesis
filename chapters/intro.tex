\chapter{绪论}
%\section{模板简介}
%测试脚注\footnote{分别编号}。
%
%\subsection{模板介绍1}
%
%\subsubsection{模板测试}
%
%\subsection{模板介绍2}
%
%\section{系统要求}
%
%\section{问题反馈}
%测试脚注\footnote{脚注2}



\section{课题背景}
随着信息技术的不断发展,如今的数据产生速度越来越快,信息的交流越来越便捷。在如今的大数据时代,人们在进行消费面临的选择十分之多,很多时候用户真正感兴趣的商品被淹没在了无尽的其他商品之中。现代的电商行业发展迅速,网络购物对于人们来说十分普遍,在网络的过程中,电商平台如何通过给用户推荐感兴趣的商品,成为了重要的研究问题。这促进了对于推荐系统的研究。国外的Amazon电商平台\cite{Linden:2003kc},pandora音乐推荐平台,netflix影音平台,youtube平台\cite{Davidson:2010hg},国内的淘宝,京东,都面临着在用户数和商品数不断增多下如果准确地推荐的问题。

在目前对于推荐系统的研究中,主要将推荐系统划分成基于内容过滤与基于协同过滤两大类的个性化推荐算法。协同过滤算法主要分为隐式因子模型和基于领域的推荐算法两大类,或者根据用户与系统的交互方式分为显式评分系统与隐式评分系统。推荐系统中主要面临着以下问题:
\begin{enumerate}
    \item  数据量大,用户多,产品多
    \item 有些应用要求实时性比较强
    \item 新用户、新产品的冷启动问题
    \item 老用户的评分过多
    \item  用户的兴趣可能是一直在变的。通常来说新的交互具有的价值更高
    \item 恶意用户对于系统的攻击问题
 \end{enumerate}
 在这些问题中,对于推荐系统算法的并行化和计算速度的提升的研究尤为重要。
 
\section{相关研究现状}
目前对于推荐系统已经存在了一定数量的研究,包括使用KMeans通过聚类方法来降低用户相似性计算的复杂度使用kMeans来进行推荐\cite{Zahra:2015ja}\cite{Kularbphettong:2014vm}\cite{Dakhel:2011dt},以及对于推荐系统中协同过滤算法的研究\cite{Shi:2014fj}\cite{Su:2009cl},推荐系统中算法的分布式和可扩展性研究\cite{Cacheda:2011bh}\cite{Narang:2011iq}\cite{Schelter:2012jg},对于推荐系统中冷启动问题的研究\cite{Liu:2014hc},以及一些综述性文章。\cite{Portugal:2015vy}
对于推荐系统在Hadoop平台上的一些研究\cite{Zhao:2010ha}\cite{Jiang:2011bp}
以及隐式反馈研究\cite{Hu:2008el}
推荐系统中的矩阵分解技术\cite{Koren:2009jg}
隐式反馈的协同过滤算法
协同过滤算法的隐式语义模型\cite{hofmann2004latent} % todo modify
以及混合混合推荐\cite{Burke:2002fy}

\section{本文的研究内容}
本文主要研究推荐系统中的协同过滤算法的可扩展性问题以及相应的算法实现。传统的推荐算法可能面临着难以分布式处理,当数据量较大时扩展性不好的问题,本文对协同过滤算法下的基于隐式因子分解模型的交替最小二乘法和基于邻域关系下的用户-用户相似关系推荐以及物品-物品相似关系推荐算法进行了分析与实现。
\section{论文结构}
论文第一章节介绍相关背景,论文第二章节介绍使用的相关技术,论文第三章节介绍我们的系统结构设计,论文第四部分分析实验结果。第五章节进行总结与展望。