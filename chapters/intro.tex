\chapter{绪论}
%\section{模板简介}
%测试脚注\footnote{分别编号}。
%
%\subsection{模板介绍1}
%
%\subsubsection{模板测试}
%
%\subsection{模板介绍2}
%
%\section{系统要求}
%
%\section{问题反馈}
%测试脚注\footnote{脚注2}



\section{课题背景}

越来越增长的数据,用户没法发现自己的兴趣,需要有推荐系统来帮助用于挖掘到自己感兴趣的东西。

基于邻域的协同过滤算法在扩展上面的问题,提出一种解决方案,并且在spark平台上进行实验完成。
 1. 数据量大,用户多,产品多
 2. 有些应用要求实时性比较强
 3. 新用户、新产品的冷启动问题
 4. 老用户的评分过多
 5. 用户的兴趣可能是一直在变的。通常来说新的交互具有的价值更高
\section{相关研究现状}
\section{本文的研究内容}
\section{论文结构}
论文第一章节介绍相关背景,论文第二章节介绍使用的相关技术,论文第三章节介绍我们的系统结构设计,论文第四部分介绍实验结果。第五章节进行总结与展望。