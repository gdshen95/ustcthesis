\chapter{绪论}
%\section{模板简介}
%测试脚注\footnote{分别编号}。
%
%\subsection{模板介绍1}
%
%\subsubsection{模板测试}
%
%\subsection{模板介绍2}
%
%\section{系统要求}
%
%\section{问题反馈}
%测试脚注\footnote{脚注2}



\section{课题背景}
随着信息技术的不断发展,如今的数据产生速度越来越快,信息的交流越来越便捷。在如今的大数据时代,人们在进行消费面临的选择十分之多,很多时候用户真正感兴趣的商品被淹没在了无尽的其他商品之中。现代的电商行业发展迅速,网络购物对于人们来说十分普遍,在网络购物的过程中,电商平台如何通过给用户推荐感兴趣的商品,成为了重要的研究问题。这促进了对于推荐系统的研究。国外的Amazon电商平台\cite{Linden:2003kc},Pandora音乐推荐平台,Netflix影音平台,YouTube平台\cite{Davidson:2010hg},国内的淘宝,京东等企业,都面临着在用户数和商品数不断增多下如何准确地推荐的问题。

在目前对于推荐系统的研究中,主要将推荐系统划分为个性化推荐系统与非个性化推荐系统,主流的研究集中于个性化推荐系统。个性化推荐算法可以划分成基于内容过滤与基于协同过滤两大类的个性化推荐算法。协同过滤算法主要分为隐式因子模型和基于邻域的推荐算法两大类,或者根据用户与系统的交互方式分为显式评分系统与隐式评分系统。对于推荐系统的研究主要基于以下问题:
\begin{enumerate}
    \item  数据总量大,用户多,产品多。
    \item 有些应用对推荐的实时性比较强。
    \item 新用户、新产品的冷启动问题。
    \item 老用户的评分过多,拖慢计算速度。
    \item  用户的兴趣是一直在变的,通常来说用户与系统新的交互具有的价值更高。
    \item 恶意用户利用推荐算法的漏洞攻击系统。
 \end{enumerate}
 在这些问题中,对于推荐系统算法的并行化和计算速度的提升的研究尤为重要。
 
\section{相关研究现状}
1994年,来自明尼苏达大学的GroupLens研究小组开始对一个名叫MovieLens的电影网站进行研究,研究的主要内容是对该网站的用户打分情况进行分析,从而得到用户可能感兴趣的电影。随后,著名电商亚马逊将推荐系统应用到了购物中,通过分析用户历史购买行为向他们推荐可能会购买的物品,从而一举将销售额提高了近三成。自此,作为一个计算机科学、信息学。社会学等多学科交叉的新兴领域,推荐系统及其技术吸引着越来越多学者的关注。

推荐系统目前被人们普遍接受的定义是由1997年Resnick和Varian提出的\cite{resnick1997recommender}。最近20年来,对于推荐系统的主要研究主要集中在两个方面:在推荐算法层面上,发明和改进推荐算法,从而能够面对新的应用场景,或者在相对应的场景下提高某些性能指标,如准确度等;另外在工程实践方面,如何使得推荐系统能够面对大数据环境下的挑战,提升推荐系统的实时性和稳定性,以及推荐系统中计算的并行化。这两个方面理论与技术的不断发展,推动者推荐系统的不断发展与进步。

对于推荐系统而言,当前的推荐算法主要可以分为,基于协同过滤的推荐,基本思想是利用用户的行为数据,基于群体智慧进行推荐;基于知识的推荐,在某些特殊领域,用户的行为数量不足,我们需要基于余弦了解的知识对于用户的喜好与需求进行预测;基于内容的推荐,该算法核心是利用分析物品的描述信息及其相关特征,通过聚类来分析出用户的感兴趣商品。

本文主要针对推荐系统中的协同过滤算法进行相关的研究,在协同过滤算法方面存在一些相关的研究如下。文献\cite{Zahra:2015ja}提出了一种在利用KMeans聚类的时候一种新的中心点选择的方案。文献\cite{Kularbphettong:2014vm}在常规的协同过滤算法中结合KMeans聚类方法并在Android平台上实现了推荐系统。文献\cite{Dakhel:2011dt}同样适用KMeans聚类结合邻居投票的方法实现了一种新的协同过滤算法。%包括使用KMeans通过聚类方法来降低用户相似性计算的复杂度和直接使用kMeans来进行推荐。
%对于推荐系统中协同过滤算法的研究
文献\cite{Shi:2014fj}\cite{Su:2009cl}对于当代基于用户商品评分矩阵的协同过滤算法总结了当前的研究现状与未来的挑战。文献\cite{Cacheda:2011bh}分析了当前协同过滤算法在可扩展与高性能的需求下的一些限制.
%推荐系统中算法的分布式和可扩展性研究
文献\cite{Narang:2011iq}主要提出了一个分布式可扩展的协同过滤算法,文献\cite{Schelter:2012jg}对于基于相似性关系的推荐算法在MapReduce计算模式下提出了可扩展的方案。文献\cite{Liu:2014hc}对于推荐系统中冷启动问题进行了研究,主要可以分为系统、用户和商品三者的冷启动问题。
文献\cite{Zhao:2010ha}对于Hadoop平台下基于用户的协同过滤算法进行了分析与实现,
%对于推荐系统在Hadoop平台上的一些研究
文献\cite{Jiang:2011bp}对于Hadoop平台下基于物品的协同过滤算法进行了扩展性方案实现。
文献\cite{Hu:2008el}对于推荐系统中的隐式反馈评分方案进行了研究。
文献\cite{Koren:2009jg}对于推荐系统中的矩阵分解技术进行了研究。
文献\cite{hofmann2004latent}对于协同过滤算法的隐式语义模型进行了研究。
文献\cite{Burke:2002fy}对于推荐系统中混合推荐算法进行了研究。

\section{本文的研究内容}
本文主要研究推荐系统中的协同过滤算法的可扩展性问题以及进行相应的算法实现。传统的推荐算法可能面临着难以分布式处理,当数据量较大时扩展性不好的问题。本文完成了两方面的工作,其一是基于交替最小二乘法完成了隐式因子模型,主要是利用ALS算法对大矩阵进行分布式分解,通过分解结果重建矩阵,利用重建的矩阵进行评分;其二的工作可以细分为两个小方面,分别对基于用户-用户和物品-物品的相似关系的邻域推荐算法进行了分布式实现,并且根据在实现过程中加入了下采样的操作,降低了邻域推荐算法的时间复杂度。
%本文对协同过滤算法下的基于交替最小二乘法的隐式因子模型和基于邻域关系下的用户-用户相似关系推荐以及物品-物品相似关系推荐算法进行了分析与实现。
\section{论文结构}
论文第一章节对于推荐系统和大数据环境下的相关背景进行了介绍;论文第二章节对采用的相关技术进行了介绍与分析,包括Spark平台的搭建与使用、推荐系统中协同过滤算法下的隐式因子模型和邻域推荐算法详细介绍与分析;论文第三章节介绍了我们的系统设计,包含数据源的选择,并行算法的实际设计;论文第四章节对于我们设计的系统进行了实现,并记录了相应的实验结果,对实验结果进行了综合分析;论文第五章节对我所做的工作进行了总结并展望了可以下一步扩展的地方。
%论文第一章节介绍相关背景,论文第二章节介绍使用的相关技术,论文第三章节介绍我们的系统与算法设计,论文第四章节分析实验结果。第五章节进行总结与展望。